\chapter{
    \textbf{CONCLUSION}
}
\justifying {
    \large
        \paragraph{} An efficient two-stage video steganography method is introduced, which is based on adaptive LSB332 and enhanced Vigenere encryption. In the first stage of the method, an efficient frame selection method based on the Vigenere encryption technique is used. This proposed frame selection algorithm is supported by the Fibonacci sequence and a three-bit key. The proposed frame selection algorithm provides more security than the sequential selection algorithms used in video steganography, and less computational complexity than chaotic frame selection algorithms at the same security level. In the second step of the proposed method, adaptive inverted LSB332 method is used to provide high-capacity data hiding. The hiding capacity of this method is compatible with increasing up to 2.6 bits per pixel. The visual quality and similarity values of stego videos are better than the classical LSB332 method, all existing LSB332 based or other similar video steganography methods.
        \paragraph{} The proposed method achieved approximately 55 dB and 50 dB PSNR values and nearly 0.99 SSIM values at 200 Kbits and 500 Kbits data capacities, respectively. The experimental analysis showed approximately 12\% increase in capacity at the same visual qualities when compared to classical LSB332-based methods and provided simpler, more efficient, secure and lower computational complexity with frame selection method. In addition, the visual quality of the proposed method is much better than other video data hiding methods, and the capacity is increased by 19\%.
        \paragraph{} The proposed data hiding technique is one of the spatial domain techniques and is easy to implement, but as it is known, it is more prone to attacks. In literature studies, it was seen that transform domain techniques are more resistant to attacks than spatial domain techniques. Also, among the transform domain techniques, Discrete Wavelet Transformation (DWT) has more advantages over Discrete Cosine Transformation, and it was observed that it makes the DWT technique preferable for embedding. According to this information, in the future, studies shall be carried out to achieve high embedding capacity, high visual quality, low complexity and robustness by using DWT for the transformation domain.
}